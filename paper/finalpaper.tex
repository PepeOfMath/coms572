%%%%%%%%%%%%%%%%%%%%%%%%%%%%%%%%%%%%%%%%%%%%%%%%%%%%%%%%%%%%%%%%%%
%%%%%%%% ICML 2015 EXAMPLE LATEX SUBMISSION FILE %%%%%%%%%%%%%%%%%
%%%%%%%%%%%%%%%%%%%%%%%%%%%%%%%%%%%%%%%%%%%%%%%%%%%%%%%%%%%%%%%%%%

% Use the following line _only_ if you're still using LaTeX 2.09.
%\documentstyle[icml2015,epsf,natbib]{article}
% If you rely on Latex2e packages, like most moden people use this:
\documentclass{article}

% use Times
\usepackage{times}
% For figures
\usepackage{graphicx} % more modern
%\usepackage{epsfig} % less modern
\usepackage{subfigure} 

% For citations
\usepackage{natbib}

% For algorithms
\usepackage{algorithm}
\usepackage{algorithmic}

% As of 2011, we use the hyperref package to produce hyperlinks in the
% resulting PDF.  If this breaks your system, please commend out the
% following usepackage line and replace \usepackage{icml2015} with
% \usepackage[nohyperref]{icml2015} above.
\usepackage{hyperref}

% Packages hyperref and algorithmic misbehave sometimes.  We can fix
% this with the following command.
\newcommand{\theHalgorithm}{\arabic{algorithm}}

% Employ the following version of the ``usepackage'' statement for
% submitting the draft version of the paper for review.  This will set
% the note in the first column to ``Under review.  Do not distribute.''
\usepackage[accepted]{icml2015} 

% Employ this version of the ``usepackage'' statement after the paper has
% been accepted, when creating the final version.  This will set the
% note in the first column to ``Proceedings of the...''
%\usepackage[accepted]{icml2015}


% The \icmltitle you define below is probably too long as a header.
% Therefore, a short form for the running title is supplied here:
\icmltitlerunning{Monte Carlo Tree Search for the Pokemon Trading Card Game}

\begin{document} 
\twocolumn[
\icmltitle{Monte Carlo Tree Search for the Pokemon Trading Card Game}

% It is OKAY to include author information, even for blind
% submissions: the style file will automatically remove it for you
% unless you've provided the [accepted] option to the icml2015
% package.
\icmlauthor{ Tyler Uhlenkamp }{ tauhlenk@iastate.edu }
\icmladdress{ Iowa State University }
\icmlauthor{ Tyler Chenhall }{ tchenhal@iastate.edu }
\icmladdress{ Iowa State University }

% You may provide any keywords that you 
% find helpful for describing your paper; these are used to populate 
% the "keywords" metadata in the PDF but will not be shown in the document
\icmlkeywords{boring formatting information, machine learning, ICML}

\vskip 0.3in
]

\begin{abstract} % Done

The Pokemon Trading Card Game is a popular turn-based card game usually played by two players. Its partially observable, stochastic, multi-agent nature combined with its complex rules, large state space, and demand for long-term planning make it an interesting target for modern artificial intelligence research efforts. An agent based on a Monte Carlo Tree Search (MCTS) algorithm with added knowledge via a heuristic function was developed and tested in games of PCG against human players and other agents. Although the MCTS agent performed well against random agents, its performance against human players was  worse than expected. Possible explanations for this performance gap are explored and suggestions for future research and technical enhancements are proposed.

\end{abstract} 

\section{Introduction} % Done
The Pokemon Trading Card game has several characteristics that make it an interesting problem in artificially intelligent agent design. The general format and objective of the game, as well as its formal environment characterization and other important characteristics are summarized below.

\subsection{Background: The Pokemon Trading Card Game} % Tyler C

You have cards and play them and stuff

\subsection{Summary: Game Environment} % Tyler C

It's partially observable and all that other junk
It's got a large branching factor
It requires long-term planning in an uncertain game tree

\section{Methodology} % Done
 
In an effort to develop an agent which could play the Pokemon Card Game competently against human players, the developers implemented a custom state representation and game logic from scratch, then created an agent based on a Monte Carlo Tree Search (MCTS) and a well-tuned heuristic function. 

\subsection{Game Logic and State Representation} % Tyler C

We had some state and some game logic functions that looked like this. 

\subsection{The Agent Program} % ME
The Monte Carlo Tree Search algorithm was chosen as the basis of the Pokemon Card Game agent because the algorithm has seen wide success in a variety of related games such as Poker, Settlers of Catan, and the "Magic: The Gathering" card game. Furthermore, agents based on planning or rule-based approaches wouldn't be robust enough to create new strategies for brand new decks. The minimax algorithm also proves to be an undesirable candidate due to the partially observable and stochastic nature of the game environment as well as the large branching factor and need for long-term planning. The MCTS algorithm, on the other hand, has been applied to similar games and can be easily adapted to partially observable and stochastic environments. It also can exhibit long-term planning and works well in environments with a large branching factor, such as Go. 

\subsubsection{Monte Carlo Tree Search Implementation Details} % ME
A full description of the MCTS algorithm and its properties is left to previous research while a summary of the implementation used in the PCG agent is provided. 

The MCTS algorithm works by building a search tree node-by-node and estimating the effectiveness of each node by simulating many games which visit that node and keeping track of the expectation of reward. 

The MCTS agent developed for the PCG agent maintains a persistent game tree structure which is saved between turns. At the beginning of a turn, the current state is located in the current game tree and set as the new root node. In this way, results of simulations in previous turns can still provide information about the usefulness of nodes still under consideration. Next, the MCTS algorithm, consisting of Selection, Expansion, Simulation, and Back-propagation stages (described below) are repeated in a loop for up to five seconds. Each iteration may add up to one new node to the search tree and simulates one play-out to the end of the game. Usually, the PCG agent was able to complete 15,000 such iterations per turn on a consumer-grade laptop computer. Finally, the child of the root node with the highest computed value is chosen as the next action. 

Typical MCTS implementations assume uniform node types, however the PCG bot employed a unique tree structure in order to account for the stochastic nature of actions at each step in game play. Namely, it is assumed that each action results in one or more possible states and the state which is actually reached is the result of random chance. Therefore, the game tree consists of alternating levels of min/max nodes and chance nodes. A min/max node seeks to maximize or minimize the value of its children, while chance nodes randomly result in one of their children.

Each iteration of the MCTS algorithm consists of four stages as noted previously: Selection, Expansion, Simulation, and Back-propagation. The implementation of each stage is described below. 

\textbf{Selection:} Starting from the root node, "optimal
 children are selected recursively until a node which is a leaf node or a node which still has un-expanded children is encountered. When selecting children, one must be careful to balance the exploitation of nodes which are known to be favorable to the current player and nodes which have not been explored as much. To accomplish this goal, the Upper Confidence Bounds formula was used. Namely, at each step the child node which maximizes the following equation was chosen: $$v_i + C \times \sqrt{\frac{\ln N}{n_i}}$$ where $v_i$ is the value of the current node, 



\subsubsection{Heuristic Function} % Tyler C
Describe the heuristic function here

\section{Results} % Tyler C
Some stuff about how we tested 

\subsection{Test Results} % Tyler C
Results vs random AI, observations vs humans

\subsection{Explanations and Future Work} % ME


\section{Conclusion} % ME


% In the unusual situation where you want a paper to appear in the
% references without citing it in the main text, use \nocite
\nocite{langley00}

\bibliography{example_paper}
\bibliographystyle{icml2015}

\end{document} 


% This document was modified from the file originally made available by
% Pat Langley and Andrea Danyluk for ICML-2K. This version was
% created by Lise Getoor and Tobias Scheffer, it was slightly modified  
% from the 2010 version by Thorsten Joachims & Johannes Fuernkranz, 
% slightly modified from the 2009 version by Kiri Wagstaff and 
% Sam Roweis's 2008 version, which is slightly modified from 
% Prasad Tadepalli's 2007 version which is a lightly 
% changed version of the previous year's version by Andrew Moore, 
% which was in turn edited from those of Kristian Kersting and 
% Codrina Lauth. Alex Smola contributed to the algorithmic style files.  
