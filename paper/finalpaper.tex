%%%%%%%%%%%%%%%%%%%%%%%%%%%%%%%%%%%%%%%%%%%%%%%%%%%%%%%%%%%%%%%%%%
%%%%%%%% ICML 2015 EXAMPLE LATEX SUBMISSION FILE %%%%%%%%%%%%%%%%%
%%%%%%%%%%%%%%%%%%%%%%%%%%%%%%%%%%%%%%%%%%%%%%%%%%%%%%%%%%%%%%%%%%

% Use the following line _only_ if you're still using LaTeX 2.09.
%\documentstyle[icml2015,epsf,natbib]{article}
% If you rely on Latex2e packages, like most moden people use this:
\documentclass{article}

% use Times
\usepackage{times}
% For figures
\usepackage{graphicx} % more modern
%\usepackage{epsfig} % less modern
\usepackage{subfigure} 

% For citations
\usepackage{natbib}

% For algorithms
\usepackage{algorithm}
\usepackage{algorithmic}

% As of 2011, we use the hyperref package to produce hyperlinks in the
% resulting PDF.  If this breaks your system, please commend out the
% following usepackage line and replace \usepackage{icml2015} with
% \usepackage[nohyperref]{icml2015} above.
\usepackage{hyperref}

% Packages hyperref and algorithmic misbehave sometimes.  We can fix
% this with the following command.
\newcommand{\theHalgorithm}{\arabic{algorithm}}

% Employ the following version of the ``usepackage'' statement for
% submitting the draft version of the paper for review.  This will set
% the note in the first column to ``Under review.  Do not distribute.''
\usepackage[accepted]{icml2015} 

% Employ this version of the ``usepackage'' statement after the paper has
% been accepted, when creating the final version.  This will set the
% note in the first column to ``Proceedings of the...''
%\usepackage[accepted]{icml2015}


% The \icmltitle you define below is probably too long as a header.
% Therefore, a short form for the running title is supplied here:
\icmltitlerunning{Monte Carlo Tree Search for the Pokemon Trading Card Game}

\begin{document} 
\twocolumn[
\icmltitle{Monte Carlo Tree Search for the Pokemon Trading Card Game}

% It is OKAY to include author information, even for blind
% submissions: the style file will automatically remove it for you
% unless you've provided the [accepted] option to the icml2015
% package.
\icmlauthor{ Tyler Uhlenkamp }{ tauhlenk@iastate.edu }
\icmladdress{ Iowa State University }
\icmlauthor{ Tyler Chenhall }{ tchenhal@iastate.edu }
\icmladdress{ Iowa State University }

% You may provide any keywords that you 
% find helpful for describing your paper; these are used to populate 
% the "keywords" metadata in the PDF but will not be shown in the document
\icmlkeywords{boring formatting information, machine learning, ICML}

\vskip 0.3in
]

\begin{abstract} 

The Pokemon Trading Card Game is a popular turn-based card game usually played by two players. Its partially observable, stochastic, multi-agent nature combined with its complex rules, large state space, and demand for long-term planning make it an interesting target for modern artificial intelligence research efforts. An agent based on a Monte Carlo Tree Search (MCTS) algorithm with added knowledge via a heuristic function was developed and tested in games of PCG against human players and other agents. Although the MCTS agent performed well against random agents, its performance against human players was  worse than expected. Possible explanations for this performance gap are explored and suggestions for future research and technical enhancements are proposed.

\end{abstract} 

\section{Introduction}
\label{Introduction}
The Pokemon Trading Card game has several characteristics that make it an interesting problem in artificially intelligent agent design. The general format and objective of the game, as well as its formal environment characterization and other important characteristics are summarized below.

\subsection{Background: The Pokemon Trading Card Game}

You have cards and play them and stuff

\subsection{Summary: Game Environment}

It's partially observable and all that other junk
It's got a large branching factor
It requires long-term planning in an uncertain game tree

\section{Methodology} 
 
In an effort to develop an agent which could play the Pokemon Card Game competently against human players, the developers implemented a custom state representation and game logic from scratch, then created an agent based on a Monte Carlo Tree Search (MCTS) and a well-tuned heuristic function. 

\subsection{Game Logic and State Representation}

We had some state and some game logic functions that looked like this. 

\subsection{The Agent Program}

The paper title should be set in 14~point bold type and centered
between two horizontal rules that are 1~point thick, with 1.0~inch
between the top rule and the top edge of the page. Capitalize the
first letter of content words and put the rest of the title in lower
case.


\subsubsection{Monte Carlo Tree Search Implementation}
The paper title should be set in 14~point bold type and centered
between two horizontal rules that are 1~point thick, with 1.0~inch
between the top rule and the top edge of the page. Capitalize the
first letter of content words and put the rest of the title in lower
case.

\subsubsection{Heuristic Function}
The paper title should be set in 14~point bold type and centered
between two horizontal rules that are 1~point thick, with 1.0~inch
between the top rule and the top edge of the page. Capitalize the
first letter of content words and put the rest of the title in lower
case.

\section{Results}
The paper title should be set in 14~point bold type and centered
between two horizontal rules that are 1~point thick, with 1.0~inch
between the top rule and the top edge of the page. Capitalize the
first letter of content words and put the rest of the title in lower
case.

\section{Conclusion}
The paper title should be set in 14~point bold type and centered
between two horizontal rules that are 1~point thick, with 1.0~inch
between the top rule and the top edge of the page. Capitalize the
first letter of content words and put the rest of the title in lower
case.
% Acknowledgements should only appear in the accepted version. 
\section*{Acknowledgments} 
 
\textbf{Do not} include acknowledgements in the initial version of
the paper submitted for blind review.

If a paper is accepted, the final camera-ready version can (and
probably should) include acknowledgements. In this case, please
place such acknowledgements in an unnumbered section at the
end of the paper. Typically, this will include thanks to reviewers
who gave useful comments, to colleagues who contributed to the ideas, 
and to funding agencies and corporate sponsors that provided financial 
support.  


% In the unusual situation where you want a paper to appear in the
% references without citing it in the main text, use \nocite
\nocite{langley00}

\bibliography{example_paper}
\bibliographystyle{icml2015}

\end{document} 


% This document was modified from the file originally made available by
% Pat Langley and Andrea Danyluk for ICML-2K. This version was
% created by Lise Getoor and Tobias Scheffer, it was slightly modified  
% from the 2010 version by Thorsten Joachims & Johannes Fuernkranz, 
% slightly modified from the 2009 version by Kiri Wagstaff and 
% Sam Roweis's 2008 version, which is slightly modified from 
% Prasad Tadepalli's 2007 version which is a lightly 
% changed version of the previous year's version by Andrew Moore, 
% which was in turn edited from those of Kristian Kersting and 
% Codrina Lauth. Alex Smola contributed to the algorithmic style files.  
